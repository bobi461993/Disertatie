\chapter{Introducere}

\paragraph{}
Procesarea de limbaj natural a reprezentat dintotdeauna una dintre marile provoc\u ari ale comunit\u a\c tiilor \c stiin\c tifice de pretutindeni, scopul final fiind realizarea unui sistem suficient de inteligent \^ inc\^ at s\u a fie capabil s\u a comunice precum omul. \^ In acest scop, \^ in anul 1950 a fost lansat testul Turing care presupune o conversa\c tie om-ma\c sin\u a \c si un interlocutor care particip\u a la aceast\u a discu\c tie, acesta nefiind capabil s\u a disting\u a omul de ma\c sin\u a. Acest test \^ inc\u a st\u a \^ in picioare dup\u a aproape 70 de ani, niciun sistem fiind capabil \^ inc\u a s\u a-l doboare. 

\paragraph{}
Recent, diverse companii de succes precum Google, Facebook, Microsoft, Apple \c si al\c tii au realizat un demers semnificativ spre dezvoltarea unor sisteme inteligente de comunica\c tie numite chatbots. La ora actual\u a, cei mai mul\c ti sunt orienta\c ti c\u atre suport tehnic, scopul lor fiind tocmai de a imita sprijinul pe care o persoan\u a real\u a \^ il poate oferi unui client. Un sistem computa\c tional inteligent poate fi folosit pentru a r\u aspunde multor cereri simultane, iar costurile de intre\c tinere sunt mici – practic, odat\u a ce sistemul este dezvoltat, este necesar\u a doar expunerea lui, de exemplu ca serviciu web. Cu c\^ at tehnologia \c si munca de cercetare \^ in aceast\u a direc\c tie progreseaz\u a, cu at\^ at sistemele de genul acesta devin din ce \^ in ce mai inteligente \c si mai apropiate de ceea ce un om este capabil s\u a ofere. Precum automatizarea industrial\u a de producere a vehiculelor reprezint\u a un standard \^ in era contemporan\u a, asisten\c tii conversa\c tionali vor deveni un standard \^ in anii ce vor urma.

\paragraph{}
Majoritatea modelelor actuale se bazeaz\u a pe oferirea unor r\u aspunsuri predefinite. Aceste tipuri de sisteme pot returna doar r\u aspunsuri existente, nefiind capabile de a genera text nou. Pe de alt\u a parte, avem sistemele bazate pe inteligen\c t\u a artificial\u a generative care, precum omul, pot emite r\u aspunsuri bazate pe experien\c te anterioare. Vorbim deci de evolu\c tia unor modele de chatbots bazate pe pattern matching c\u atre unele bazate pe modele generative. \^ In ultimii ani, ramura inteligen\c tei artificiale numit\u a Machine Learning  (\^ inv\u a\c tare automat\u a) a luat amploare \^ in acest\u a direc\c tie prin curentul numit Deep Learning. Acest curent a produs o mul\c time de rezultate spectaculoase at\^ at \^ in direc\c tia proces\u arii naturale de limbaj c\^ at \c si a proces\u arii de imagini. 

\paragraph{}
Rolul principal al unui chatbot \^ il reprezint\u a capabilitatea de a ``\^ intelege`` informa\c tia primit\u a de la o persoan\u a pentru a produce un raspuns c\^ at mai plauzibil. \^ Ins\u a cum proceseaz\u a un calculator o limb\u a? Pentru a r\u aspunde la aceast\u a \^ intrebare, se va face o analogie la cum \^ inva\c t\u a un om o limb\u a. Porne\c ste de la anumite cuvinte de baz\u a iar apoi pe baza acestora, \^ inva\c t\u a cuvinte tot mai complexe. \^ Incepe s\u a creeze fraze prin care leag\u a aceste cuvinte precum \c si gramatica specific\u a limbajului. Practic, totul se bazeaz\u a pe o anumit\u a experien\c t\u a cumulativ\u a. Conversa\c tia devine astfel o modalitate ce faciliteaz\u a \c si impulsioneaz\u a deprinderea limbajului: omul este pus \^ in fa\c ta unor contexte de utilizare, fenomen ce \^ int\u are\c ste deprinderea de utilizare a cuvintelor sau expresiior individuale. Fiecare cuv\^ ant nou \^ inv\u a\c tat reprezint\u a o experien\c t\u a c\u atre \^ inv\u a\c tarea \^ in continuare a limbajului. \^ Inc\u a un aspect reprezentativ uman este capabilitatea de a \^ intre\c tine o conversa\c tie pe termen lung cu un alt participant \^ in intermediul unui context, lucru dificil de capturat pentru un sistem artificial.

\paragraph{}
Abord\u arile de chatbot actuale \^ i\c si propun imitarea acestei modalit\u a\c ti de \^ inv\u a\c tare pentru un sistem computa\c tional. Modelarea curent\u a cea mai eficient\u a pentru o astfel de problem\u a este oferit\u a de c\u atre Deep Learning. Folosind arhitecturi de re\c tele neurale artificiale mult mai complexe dec\^ at cele shallow de la \^ inceputul anilor 2000 \c si plec\^ and de la seturi de instruire masive - \^ in cazul de fa\c t\u a, corpusuri de text cu c\^ at mai multe fraze \^ in limba dorit\u a – se pot crea modele generative care pot produce continuarea unor propozi\c tii, fraze etc. Ideea din spatele abord\u arii Deep Learning este c\u a sistemul devine mai performant pe m\u asur\u a ce volumul de date cre\c ste. Pentru a procesa o asemenea cantitate de date de instruire este nevoie de putere computa\c tional\u a pe m\u asur\u a. Asta presupune pe scurt, capacitate hardware. Milioanele de calcule efectuate pentru modelarea unui limbaj de c\u atre sistem nu permit folosirea unui procesor, chiar multicore de ultim\u a genera\c tie, deoarece este considerat a duce la sugrumarea procesului de instruire.

\paragraph{}
\^ In locul microprocesoarelor se prefer\u a folosirea pl\u acilor grafice (GPU). Ini\c tial, acestea au fost dezvoltate pentru rulare rapid\u a de jocuri, dar poten\c tialul lor a fost rapid intuit \c si exploatat prin programare paralel\u a. Deoarece placa video con\c tine mult mai multe nuclee de procesare (c\^ ateva mii, comparate cu cele 4-8 nuclee tradi\c tionale dintr-un microprocesor actual), este preferat\u a programarea \c si rularea modelelor computa\c tionale pe GPU. \^ In ultimii ani au dezvoltate o multitudine de biblioteci care faciliteaz\u a unui programator dezvoltarea de aplica\c tii de Machine Learning pe GPU: Tensorflow, Theano, Caffe, Keras etc.

\section{Motiva\c tia alegerii temei}

\paragraph{}
Testul Turing reprezint\u a o provocare pe care mul\c ti cercet\u atori din acest domeniu o abordeaz\u a constant cu noi idei tot mai inovative, av\^ and din ce \^ in ce mai mult succes \^ in ultimii ani. Chiar \c si la nivelul unui model generativ de juc\u arie, acest subiect reprezint\u a unul dintre cele mai fierbin\c ti topicuri la ora actual\u a pentru comunitatea Deep Learning. Acest fapt, precum \c si cercetarea ideilor state of the art folosite \^ in sistemele actuale reprezint\u a factorul motiva\c tional principal \^ in alegerea temei de cercetare.

\section{Structura lucr\u arii}

\paragraph{}
TODO


\section{Recunoa\c stere}

\paragraph{}
Mul\c tumiri Universit\u a\c tii Transilvania din Bra\c sov care a finan\c tat acest proiect \^ in scopul achizi\c tion\u arii de hardware necesar pentru realizarea acestuia.
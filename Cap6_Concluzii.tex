\chapter{Concluzii}

\paragraph{}
\^ In aceast\u a lucrare s-a prezentat un model generativ pentru procesare natural\u a de limbaj, capabil s\u a \^ intre\c tin\u a contextul discu\c tiei folosind o arhitectur\u a de re\c tea neural\u a considerat\u a state of the art. Folosind un set de date care con\c tine replici din filme, s-a ar\u atat c\u a este posibil\u a modelarea unui sistem care poate produce r\u aspunsuri open-topic generative. De\c si rezultatele sunt departe de a oferi impresia unui partener uman angajat \^ intr-o conversa\c tie, acest experiment de cercetare realizeaz\u a un pas important c\u atre dezvoltarea unor modele lingvistice cu adev\u arat inteligente. 

\paragraph{}
Dac\u a prezentul este \^ inc\u a sub semnul \^ intreb\u arii \^ in privin\c ta acestor sisteme conversa\c tionale, eforturile uria\c se investite de c\u atre marile grupuri de cercetare ale lumii nu pot dec\^ at s\u a dovedeasc\u a interesul major pentru aceast\u a arie, rezultatele urm\^ and s\u a apar\u a f\u ar\u a \^ indoial\u a \^ in anii ce vor urma, pentru ca \^ in final un astfel de sistem s\u a doboare testul vechi de aproape 70 de ani.